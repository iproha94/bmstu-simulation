\chapter{Конструкторский раздел}

\textbf{Алгоритмическая последовательность случайных чисел} создается встроеммным в язык генератором случайных чисел. 

Примечание: метод Math.random() не предоставляет криптографически стойкие случайные числа. Не используйте его ни для чего, связанного с безопасностью. Вместо него используйте Web Crypto API (API криптографии в вебе) и более точный метод window.crypto.getRandomValues().

\textbf{Табличная последовательность случайных чисел} создается с помощью $/dev/urandom$ символьным устройством ОС Linux \cite{habrahabr-hard-rand} 

\textbf{Критерии проверки ПСЧ на случайность}:
\begin{itemize}
	\item частотный побитовый тест
	\item тест на одинаковые идущие подряд биты.
\end{itemize}

В связи с тем, что тесты поразрядные, то числа нужно генерировать с учетом старших нулей. Т.е. если формируем трехразрядные числа, то сгенерирванное число 2 (10 в двоичной системе), нужно подать на тест в виде "010".  

В программе реализовано \textbf{два режима генерации чисел} алгоритмичеким способом.

\begin{itemize}
	\item десятичный - (числа [0, 9], [10, 99], [100, 999]). Данная последовательность обречена на провал в данных тестах, так как колличество "0" и "1" в числах из данных диапазонах в двоичном ввиде будет смещенным.

	\item двоичный - при нем генерируются числа в диапазонах (числа в двоичном ввиде) [0, 1], [00, 11], [000, 111].
\end{itemize}

\textbf{Частотный побитовый тест}
Принимаем каждую «1» за +1, а каждый «0» за -1 и считаем сумму по всей последовательности.
Очевидно, что чем более случайна последовательность, тем ближе это соотношение к 1. Данный тест оценивает, насколько это соотношение близко к 1. Вычисляем статистику, затем вычисляем P-значение через дополнительную функцию ошибок. Если результат > 0.01, то значит наша последовательность прошла тест. Рекомендуется тестировать последовательности длиной не менее 100 бит. \cite{habrahabr-nist} 

\textbf{Тест на одинаковые идущие подряд биты}
В тесте ищутся все последовательности одинаковых битов, а затем анализируется, насколько количество и размеры этих последовательностей соответствуют количеству и размерам истинно случайной последовательности. Смысл в том, что если смена 0 на 1 (и обратно) происходит слишом редко, то такая последовательность «не тянет» на случайную. \cite{habrahabr-nist} 




