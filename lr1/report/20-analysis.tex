\chapter{Теоретический раздел}

\textbf{Цель} данной работы - найти среднее относительное время прибывания системы в каждом состоянии.

Данная система массового обслуживания \cite{mathhelpplanet-smo} $S$ работает в стационарном режиме.

Для решения данной задачи необходимо от заданной матрицы интенсивностей перехода из состояния в состояние перейти к уравнениям Колмогорова.

\textbf{Правила составления уравнений Колмогорова} \cite{mathhelpplanet-kolmogorov}:
\begin{itemize}
	\item В левой части каждого из них стоит производная вероятности i-го состояния.
	\item В правой части — сумма произведений вероятностей всех состояний (из которых идут стрелки в данное состояние) на интенсивности соответствующих потоков событий, минус суммарная интенсивность всех потоков, выводящих систему из данного состояния, умноженная на вероятность данного (i-го состояния).
\end{itemize}

Уравнения Колмогорова дают возможность найти все вероятности состояний как функции времени. Особый интерес представляют вероятности системы $p_i(t)$ в предельном стационарном режиме, т.е. при $t \to \infty $, которые называются \textbf{предельными (или финальными) вероятностями состояний}.

Предельная вероятность состояния $S_i$ имеет четкий смысл: она показывает \textbf{среднее относительное время} пребывания системы в этом состоянии.

Так как предельные вероятности постоянны, то, заменяя в уравнениях Колмогорова их производные нулевыми значениями, получим \textbf{систему линейных алгебраических уравнений}, описывающих стационарный режим.

В полученной системе независимых уравнений на единицу меньше общего числа уравнений. Поэтому для решения системы необходимо добавить \textbf{уравнение нормировки} (сумма вероятностей всех состояний равна единице).



