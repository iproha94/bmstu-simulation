\chapter{Аналитический раздел}

\textbf{Цель} данной работы - cмоделировать систему, состоящую из генератора, источника информации, блока памяти и обслуживающего аппарата. 

Закон генерации заявок - равномерный (параметры настраиваются и варьируются). 

В ОА закон распределение Пуассона. 

Определить оптимальный размер буферной памяти, т.е. ту длину, при которой ни одно сообщение необработанным не исчезает (т.е. нет отказа) 

Должна быть возможность возвращения заявки в очередь после ее обработки. 

Управляющую программу имитационной модели реализовать двумя подходами:
\begin{itemize}
	\item событийный
	\item $\Delta t$ (ориентирован на действие)
\end{itemize}